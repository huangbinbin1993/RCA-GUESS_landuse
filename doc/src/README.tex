\documentclass{article}

\usepackage{verbatim}
\usepackage{rotating}

\setlength{\parindent}{0cm}

\usepackage[ps2pdf,colorlinks=true,a4paper=true]{hyperref}
\hypersetup{
pdfauthor = {Rossby Centre, SMHI},
pdftitle = {Guide to the Rossby Centre Regional Climate Model RCA},
pdfsubject = {report},
pdfcreator = {LaTeX with hyperref package},
pdfproducer = {dvips + ps2pdf}}


\title{Guide to the Rossby Centre Regional Climate Model RCA} 
\author{Marco Kupiainen \and Patrick Samuelsson \and Ulf Hansson}

\begin{document}
\maketitle
\tableofcontents  
\newpage

\setlength{\parskip}{2mm}

\section{Quickstart}

\subsection{Check out RCA}

Check out the (trunk) code using:
\begin{verbatim}
svn co svn+ssh://$USER@gimle.nsc.liu.se/home/rossby/svn/rca_repository/rca/trunk SRC_DIR
\end{verbatim}
where \verb+$USER+ refers to your NSC user name and
\verb+SRC_DIR+ is the directory where RCA will be created. 

Variables referred to in this text are:

\begin{tabular}{|l|l|}
\hline 
Name & Explanation \\
\hline
\hline
\verb+SRC_DIR+ & where the source code of RCA is located \\
\verb+WORK_DIR+ & location of result output (earlier known as \verb+WORK_OUT+)\\
\verb+ARCH+ & the computer RCA is running on \\
\verb+DOMAIN+ & the name of the computational domain \\
\hline
\end{tabular}



If you feel confident or just do not need to know any details in the
compilation process, you should try this:
\subsection{Compile RCA}
\begin{itemize}
  \item Make sure you have the adequate compilers, modules etc. loaded/available.
        On gimle e.g. \verb+impi+, \verb+ifort+, \verb+icc+.
  \item Set the \verb+ARCH+ variable in your shell e.g. \verb+export ARCH=gimle+ on gimle, 
    (or if you don't want to set ARCH as an environment variable you can use \verb+make ARCH=gimle+). 
  \item Make sure that \verb+SRC_DIR/config/config.ARCH+ has the expected settings.
        E.g. specification of compiler to use, optimization level, links to MPI-directories. 
        See Section \ref{sec:configconf}.
  \item Make sure that \verb+SRC_DIR/config/definitions.global+ has the expected settings.
        E.g. \verb+-DDEBUG+, \verb+-DMPI_SRC+.
        See Section \ref{sec:configdef}.
  \item type \verb+make+ in \verb+SRC_DIR+, if you defined \verb+ARCH+ or else \verb+make ARCH=gimle+ on gimle.
  \item the executable \verb+rca.x+ should be
    located in \verb+SRC_DIR/ARCH/bin+.
  \item if all did NOT go well (the authors have never experienced,
    but...) the compiler will tell what is wrong immediately and stop
    compilation (a compilation error).
\end{itemize}

\subsection{Run RCA}

In order to ensure that the automatic arrays in \verb+phys+ are
allocated \verb+ulimit -s unlimited+ must be issued before
executing RCA. A good idea is to place this command in your
login script (e.g. \verb+~/.bashrc+) or in your runscript if you
use a runscript to execute the code (see Section \ref{sec:runscript}).

Three namelist files, \verb+namelists.dat+, \verb+namelists_namppp.dat+ and \verb+gcmpaths.[nsc,pdc,userd]+, are needed to run RCA.
Depedning on how you launch RCA the namelist files should be placed either in your \verb+WORK_DIR+
or where your runscript is located.
\verb+namelists.dat+ should be copied from an appropiate sub-directory under
\verb+SRC_DIR/reference_setups/domains+.
\verb+namelists_namppp.dat+ should be generated using \verb+SRC_DIR/tools/prepare_namelists_namppp.sh+
with an appropiate input file from \verb+SRC_DIR/reference_setups/output_variables+

A runscript for the specific computer/filesystem \verb+gimle+
(\verb+tools/runScriptGimle+) is also provided. Copy also this to e.g. your \verb+WORK_DIR+.  

To submit a job (on \verb+gimle+) go to the directory where your runscript is located and do \verb+sbatch -p nehalem runScriptGimle+.

If you wish to run the job interactively, DO NOT RUN the runscript.

Instead go into interactive mode (e.g. 8 nodes for 8 hours) 
\begin{verbatim}
interactive -N 8  -t 8:00:00
\end{verbatim}
and run RCA by 
\begin{verbatim}
mpprun [--nranks=max(64 (if running totalview))] [--totalview] SRC_DIR/$ARCH/bin/rca.x
\end{verbatim}
This is an easy way to run RCA in a varied number of cores and also possibly through Totalview.
The results, \verb+fc+-files etc, will end up in
\verb+WORK_DIR+. (Please note that one cannot rerun a simulation
without 
removing old \verb+fc+-files. Otherwise the job crashes with an odd(!)
explanation that JPLREC is too small.) 

\section{Configuration of RCA}
\label{sec:config}

\subsection{config/definitions.global}
\label{sec:configdef}

The RCA code can be configured to be different programs via
different 'define'-options, described below. By invoking one or
several of these definitions a different executable is generated. Some
of the options are not valid simultaneously, but this is not checked
for in any way, so it is up to the user to check that different
options can co-exist.

The define options are:
\begin{description} 
\item[-DDEBUG] (optional) Print verbose information written to std\_out.
\item[-DLITTLE\_ENDIAN] (machine dependent) 
\item[-DMPI\_SRC] (for parallell systems) 
\item[-DUSING\_NETCDF] (for restart files in NetCDF format) 
\item[-DOASIS4] For coupling with the ocean model RCO.
\end{description}

The sources must be recompiled after adding or removing a
definition. Recompilation is simply done by \verb+make+, since the
file \verb+definitions.global+ that contains all definition is a
dependency for all source files. If however a change is made to

\subsection{config/config.ARCH}
\label{sec:configconf}

If a change is made to a \verb+config.ARCH+ file to change e.g. the degree of optimization a
\verb+make clean+ (which cleans away all object files) is necessary
before recompilation. 


\section{Runscript}
\label{sec:runscript}

The runscript is very simple. It contains \verb+SBATCH+-options for the queueing system and
a definition of to the \verb+WORK_DIR+. The runscript assumes that proper namelists are available
in the directory where the runscript is launched. The runscript also copies an analysing script
to \verb+WORK_DIR+ which can be submitted during or after a simulation for postprocessing and
analysis of results.

\section{gcmpaths.[*]}
\subsection{rout\_path}
\begin{tabular}{|l|l|l|l|l|}
  \hline
  varible & type[(length)] & explanation & default \\
  \hline
  \hline
  routing\_path & char(132) & & '' \\
  \hline
\end{tabular}

\subsection{climyears}
\begin{tabular}{|l|l|l|}
  \hline
  varible & type[(length)] & explanation \\
  \hline
  \hline
  climyears\_path & char(*) & Who knows \\
 \hline
\end{tabular}

\subsection{gtopo30}
\begin{tabular}{|l|l|l|}
  \hline
  varible & type[(length)] & explanation \\
  \hline
  \hline
  gtopo30\_path & char(*) & path to where gtopo30 database resides \\
 \hline
\end{tabular}

\subsection{ecoclimap}
\begin{tabular}{|l|l|l|}
  \hline
  varible & type[(length)] & explanation \\
  \hline
  \hline
  ecopath & char(*) & path to where Ecoclimap database resides \\
 \hline
\end{tabular}

\subsection{ecice}
\begin{tabular}{|l|l|l|}
  \hline
  varible & type[(length)] & explanation \\
  \hline
  \hline
  ecice\_path & char(*) & path to where ECMWF ice global data resides \\
 \hline
\end{tabular}

\subsection{global\_clim}
\begin{tabular}{|l|l|l|}
  \hline
  varible & type[(length)] & explanation \\
  \hline
  \hline
  global\_clim\_path & char(*) & path to where GCM lower global data resides \\
 \hline
\end{tabular}

\subsection{bound}
\begin{tabular}{|l|l|l|}
  \hline
  varible & type[(length)] & explanation \\
  \hline
  \hline
  bound\_path & char(*) & path to where lateral boundary data resides \\
 \hline
\end{tabular}

\subsection{ocean}
\begin{tabular}{|l|l|l|}
  \hline
  varible & type[(length)] & explanation \\
  \hline
  \hline
  ocean\_path & char(*) & Who knows \\
 \hline
\end{tabular}

\subsection{orog}
\begin{tabular}{|l|l|l|}
  \hline
  varible & type[(length)] & explanation \\
  \hline
  \hline
  orog\_path & char(*) & Who knows \\
 \hline
\end{tabular}

\subsection{ghg}
\begin{tabular}{|l|l|l|}
  \hline
  varible & type[(length)] & explanation \\
  \hline
  \hline
  ghg\_path & char(*) & Who knows \\
 \hline
\end{tabular}
\section{namelists.dat}
\label{sec:namelist}

Here follows a documentation of all variables that are possible in
the different namelists in \verb+namelists.dat+ as read by \verb+RCA+. Some variables are
mandatory, others are optional and are marked with \emph{emphasized} text.
\subsection{institute}
\begin{tabular}{|l|l|l|l|l|}
  \hline
  varible & type[(length)] & explanation & default \\
  \hline
  \hline
  inst & char & 'nsc','pdc' or 'user' & no default \\
  \hline
\end{tabular}
\subsection{namconfig}
\begin{tabular}{|l|l|l|l|l|}
  \hline
  varible & type[(length)] & explanation & default \\
  \hline
  \hline
  \emph{use\_oasis} & bool & use oasis & .false.\\
  \emph{use\_guess} & bool & use LPJ-guess & .false. \\
  \emph{use\_routing} & bool & use routing & .false. \\
  \emph{use\_match} & bool & use MATCH & .false. (for future use) \\
  \hline
\end{tabular}
\subsection{namrouting}
The variables only have meaning if \verb+use_routing=.true.+

\noindent
\begin{tabular}{|l|l|l|l|l|}
  \hline
  varible & type[(length)] & explanation & default \\
  \hline
  \hline
  klon\_bound & int & & - \\
  klat\_bound & int & & - \\
  \emph{ntile} & int & & 1 \\
  \emph{dimstate} & int & & 1000 \\
  \emph{acctime} & real & & 86400 \\
  \hline
\end{tabular}


\subsection{namphys}
\begin{tabular}{|l|l|l|l|l|}
  \hline
  varible & type[(length)] & explanation & default \\
  \hline
  \hline
  \emph{latvap} & real &  & 2.5003e6\\
  \emph{rair} & real &  & 2.8704e2 \\
  \emph{cpair}& real &  & 1.0046e3 \\
  \emph{ccpq}& real &  & 0.8593 \\
  \emph{epsilo} & real &  & 0.622 \\
  \emph{gravit} & real & gravity acceleration constant & 9.80665 \\
  \emph{tmelt} & real &  & 273.16 \\
  \emph{latice} & real &  & 3.336e5 \\ 
  \emph{rhos} & real &  & 1.3e3 \\
  \emph{rhoh2o} & real &  & 1.0e3 \\
  \emph{solar} & real &  & 1.367E3 \\
  \emph{stebol} & real &  & 5.67e-8 \\
  \emph{carman} & real &  & 0.40 \\
  \emph{rearth} & real & radius of earth & 6.37e6  \\
  \emph{lbkf} & bool & Use BKF convection parametrization vs. KF & .true. \\
  \emph{maximum\_random} & bool &   & .false. \\ 
  \emph{lcarb} & bool & Use soil organic carbon  & .true. \\ 
  \emph{lmulch} & bool & Use vegetation mulch effect  & .true. \\ 
\hline
\end{tabular}
Note: Current experience is that Bechtold Kain-Fritsch should no be used for South Amercian domains.

\subsection{namvar}
\begin{tabular}{|l|l|l|l|l|}
  \hline
  varible & type[(length)] & explanation & default \\
  \hline
  \hline
  \emph{iacdg2} & int & number of 2D fields to be accumulated & 13 \\
  \emph{iacdg} & int & number of 3D fields to be accumulated & 1\\
  \emph{ksvar} & int & number of extra scalar variables, e.g. TKE & 1 \\
  \hline
\end{tabular}

\subsection{namprc}
\begin{tabular}{|l|l|l|l|l|}
  \hline
  varible & type[(length)] & explanation & default \\
  \hline
  \hline
  \emph{nhorph} & int & number of grid points in each physics subarea & 500  \\
  \emph{lserial} & bool & run physics in serial mode &  .true. \\
  \hline
\end{tabular}

\subsection{namprocessor}
\begin{tabular}{|l|l|l|l|l|}
  \hline
  varible & type[(length)] & explanation & default \\
  \hline
  \hline
  \emph{nprocx} & int & number of MPI-ranks to use in longitudal direction  & computed by rca \\
   \emph{nprocy}  & int & number of MPI-ranks to use in latitudal direction &  computed by rca\\
  \hline
\end{tabular}

\subsection{namgcm}
\begin{tabular}{|l|l|l|l|l|}
  \hline
  varible & type[(length)] & explanation & default\\
  \hline
  \hline
  \emph{lecmwf} & bool & ECMWF (ERA40 or ERA Interim) data on the boundaries & .false. \\
  \emph{lhadley} & bool & HadCM data on the boundaries &  .false.\\          
  \emph{lipsl}   & bool & IPSL data on the boundaries & false.   \\       
  \emph{lecham2} & bool & ECHAM2 data on the boundaries & .false.     \\     
  \emph{lecham5} & bool & ECHAM5 data on the boundaries & .false.       \\   
  \emph{lccsm}   & bool & CCSM data on the boundaries & .false.         \\ 
  \emph{lifs}    & bool & IFS data on the boundaries & .false. \\
  \emph{lcanesm2} & bool & CanESM2 data on the boundaries & .false. \\
  \hline
\end{tabular}

\subsection{scenario}
\begin{tabular}{|l|l|l|l|l|}
  \hline
  varible & type[(length)] & explanation & default\\
  \hline
  \hline
  \emph{lrcp} & bool & use CO2 specification from control RCP period  &.false. \\
  \emph{lrcp45} & bool & use CO2 specification from RCP 4.5  &.false. \\
  \emph{lrcp85} & bool & use CO2 specification from RCP 8.5  &.false. \\
  \emph{lsres} & bool & use CO2 specification from control SRES period  &.false. \\
  \emph{lsresa1b} & bool & use CO2 specification from SRES A1B  &.false. \\
  \emph{lsresa2} & bool & use CO2 specification from SRES A2  &.false. \\
  \emph{lsresb2} & bool & use CO2 specification from SRES B2  &.false. \\
  \emph{lsresb1} & bool & use CO2 specification from SRES B1  &.false. \\
  \hline
\end{tabular}

\subsection{namrestart}
\begin{tabular}{|l|l|l|l|l|}
  \hline
  varible & type[(length)] & explanation & default\\
  \hline
  \hline
  \emph{doRestart} & bool & start from a dumpFile & .false. \\
  \emph{ntimesteps} & int & number of timesteps to take before dump & -1 \\
  \emph{stopYear} & int & what year simulation ends & 1990 \\
  \emph{stopMonth} & int & what month simulation ends & 1 \\
  \emph{stopDay} & int & what day simulation ends & 1 \\
  \emph{stopHour} & int & what hour simulation ends & 0 \\
  \emph{stopMin} & int & what minute simulation ends & 0 \\
  \emph{stopSec} & int & what second simulation ends & 0 \\

  \emph{reyear} & int & what year to run from & 1990 \\
  \emph{remonth} & int & what month to run from & 01 \\
  \emph{reday} & int & what day to run from & 01 \\
  \emph{rehour} & int & what hour to run from & 00\\
  \emph{remin} & int & what minute to run from & 00\\
  \emph{resec} & int & what second to run from & 00\\
  \emph{monthly} & bool & do monthly dumps/else yearly & .false. \\
  \hline
\end{tabular}
\subsection{namdiffh}
\begin{tabular}{|l|l|l|l|l|}
  \hline
  varible & type[(length)] & explanation & default\\
  \hline
  \hline
  \emph{nldifu} & bool & impl. hor. diffusion for u & true \\
  \emph{ndifu} & int &  order of impl. hor. diffusion for u & 6\\   
  \emph{cdifuin} & real(klev) &  horizontal diffusion constants for u & 1.0 \\
  \emph{nldifv} & bool & impl. hor. diffusion for v & true \\
  \emph{ndifv} & int &  order of impl. hor. diffusion for v & 6\\   
  \emph{cdifvin} & real(klev) &  horizontal diffusion constants for v & 1.0 \\
  \emph{nldift} & bool &  impl. hor. diffusion for T & true \\
  \emph{ndift} & int & order of impl. hor. diffusion for T & 6 \\     
  \emph{cdiftin} & real(klev)  & horizontal diffusion constants for T & 1.0 \\
  \emph{nldifq} & bool & impl. hor. diffusion for q  & true \\
  \emph{ndifq} & int & order of impl. hor. diffusion for q & 6 \\     
  \emph{cdifqin} & real(klev) & horizontal diffusion constants for q & 1.0 \\
  \emph{nldifs} & bool &  use impl. hor. diffusion for cloud water  & true \\
  \emph{ndifs} & int & order of impl. hor. diffusion for cloud water & 6 \\     
  \emph{cdifsin} & real(klev) &  horizontal diffusion constants for cloud water & 1.0 \\
  \emph{nldifxin} & bool(ksvar) & impl. hor. diffusion for extra passive scalars & true \\
  \emph{ndifxin} & real(ksvar) & order of impl. hor. diffusion for extra passive scalars & 6 \\
  \emph{cdifxin} & real(ksvar,klev) & horizontal diffusion constants for extra passive scalars & 1.0 \\%24*1.,
  \emph{nlhdif} & bool &  use explicit hor. diffusion in dynamics & false \\
  ak4 & real & diffusion coefficient for 2nd order hor. diffu computed statistics & 1.0e+14  \\
  \emph{ak4levin} & real(nc*klev) &  (nc=numberOfComponents in state vector) & (1.0*klev*4) rest 0.0 \\
  \hline 
\end{tabular}

\subsection{namsl}
\begin{tabular}{|l|l|l|l|l|}
  \hline
  varible & type[(length)] & explanation & default\\
  \hline
  \hline
  \emph{nlslan} & bool & use s-l advection & true (must always be true)\\
  \emph{nslpqi} & int &  number of iterations for calculation of displacement  & 1 \\ 
  \emph{nslinc} & int & type of interpolation at the midpoint (not used) & 4 \\  
  \emph{nslind} & int & type of interpolation at the departure point  & 4\\  
  \emph{nslint} & int(100) & list of interpolation types for each iteration & 4*100\\
  \emph{nlsl3d} & bool & use 3-dim. semi-lagrangian advection  & .true. \\
  \emph{epsg} & real & coefficient fot the gravity wave damper  & 0.2 \\
  \emph{epsn} & real & coefficient fot the gravity wave damper  & 0.2 \\
  \emph{nslext} & int &  order of time-extrapolation from $t^n$ to $t^{n+1/2}$ & 2 \\  
  \emph{nlitrh} & bool & true for iterative Helmholtz-solver & .false. \\
  \emph{aerrih} & real & abs. error tol. for iterative Helmholtz-solver & 1.0e-13\\
  \emph{rerrih} & real & rel. error tol. for iterative Helmholtz-solver & 1.0e-8\\
  \emph{nityph} & int & iteration type for iterative Helmholtz solve & 2 \\
  \emph{nptyph} & int & preconditioning type for iterative Helmholtz-solver & 1 \\
  \emph{dynamic\_halo} & bool & use dynamic halo in SL & true \\
  \emph{khalo} & int & if not dynamic halo then specifies halo width & 10 \\
  \emph{safety\_factor} & real & used in dynamic khalo computation & 0.7 \\
\hline 
\end{tabular}

\subsection{nambc}
\begin{tabular}{|l|l|l|l|l|}
  \hline
  varible & type[(length)] & explanation & default \\
  \hline
  \hline
  \emph{npbpts} & int & number of passive boundary points & 2 \\
  \emph{nbdpts} & int & number of gridpoints in the boundary relaxation   & 8 \\
		&     & zone for the boundary relaxation \\
  \emph{nltanh} & bool & use tanh-shape boundary relaxation function  & true \\
  \hline
\end{tabular}

\subsection{namrun}
\begin{tabular}{|l|l|p{5cm}|l|l|}
  \hline
  varible & type[(length)] & explanation & default\\
  \hline
  \hline
  dtime & int &   timestep in seconds & \\
  \emph{nlsimp} & bool & use semi-implicit sheme & true \\
  \emph{nlphys} & bool & use physical paramerization & true \\
  nlstat & bool &  compute and print of statics  & true \\
  nltvir & bool & use virtual temperature in dynamics & true \\
  nlhumc & bool & check of critical humidity for input data & true \\
  \emph{nltcrf} & bool & use correction for hor. diff. of T and humidity along pseud press. lev.  & true \\
  \emph{dtphys} & real & timestep for physics in seconds    & real(ndtime) \\
  \emph{dtvdif} & real & imestep for vertical diffusion in seconds & real(ndtime)  \\
  \emph{timesu} & real & spinup time in seconds &  real(2*ndtime) \\
  \emph{nldynvd} & bool & use dynamic tendency used in the vertical diffusion scheme  & false \\
  nwmosvin &  int &  WMO-CODE FOR EXTRA SCALARS & \\
  \emph{sit0} & real & reference temperature (Kelvin)  & 300.0 \\
  sip0 & real & reference surface presssure (pa)  & 101320.0 \\
  \emph{nlusug} &  bool &   ??? & false \\
  \emph{month\_file} & bool & One fc-file per month instead of one for each output-time interval. & true \\
  \hline
\end{tabular}



\subsection{namtun}
\begin{tabular}{|l|l|l|l|l|}
  \hline
  varible & type[(length)] & explanation & default\\
  \hline
  \hline
  \emph{acrit}   & real &   threshold for critical humidity & 1.00 \\
  \emph{tseafr}  & real &   freezing temperature for salty sea water & 271.15 \\
 \hline
\end{tabular}

\subsection{nampos}
\begin{tabular}{|l|l|l|l|l|}
  \hline
  varible & type[(length)] & explanation & default\\
  \hline
  \hline
  \emph{lphys} & bool &      & true \\
  \emph{lomega} & bool &  compute omegas: omh and omf & true \\
  \emph{iminpp} & int & 1:st i-index of output & 1:st inner point \\ 	
  \emph{jminpp} & int & 1:st j-index of output & 1:st inner point \\ 	
  \emph{npplon} & int & last i-index of output & last inner point \\ 	
  \emph{npplat} & int & last j-index of output & last inner point \\ 	
  \emph{linner} & bool & if .t. then output inner domain else whole domain & .true. \\
 \hline
\end{tabular}

\subsection{domain}
\begin{tabular}{|l|l|l|}
  \hline
  varible & type[(length)] & explanation \\
  \hline
  \hline
  klon\_global & int & number of gridpoint in longitudal direction \\ 
  klat\_global & int & number of gridpoints in latitudal direction \\
  klev\_global & int & number of vertical levels \\
  dlon         & real & longitudal grid spacing on the model grid   (uniform) \\
  dlat         & real & latitudal grid spacing on the model grid  (uniform) \\
  south        & real & latitudal coordinate of the model grid corner \\
  west         & real & longitudal coordinate of the model grid corner \\
  polon        & real & the longitudal coordinate of the projected   south pole \\
  polat        & real & the latitudal coordinate of the projected  south pole \\
 \hline
\end{tabular}



\subsection{namtsf}
\begin{tabular}{|l|l|l|l|}
  \hline
  varible & type[(length)] & explanation & default\\
  \hline
  \hline
 \emph{iunita} & int &  unit number of output fiel & 97\\
 \emph{imodea} & int & TSF (0) or BUFR (1) output format & 0\\
 \emph{ifreqa} & int & Sample frequency for ML fields & 1\\
 \emph{zlona} & real(jpnts) & Longitude of points for ML fields & 0.0\\
 \emph{zlata} & real(jpnts) & Latitude of points for ML fields & 0.0\\
 \emph{landa} & int(jpnts) & Index for nearest/land/sea point & 0\\
 \emph{iunitb} & int & just here to maintain old namelist & \\
 \emph{imodeb} & int & TSF (0) or BUFR (1) output format & \\
 \emph{ifreqb} & int & Sample frequency for SL fields & 1\\
 \emph{zlonb} & real(jpnts) & Longitude of points for SL fields & 0.0\\  
 \emph{zlatb} & real(jpnts) &  Latitude of points for SL fields & 0.0 \\
 \emph{landb} & int(jpnts) & Index for nearest/land/sea point & 0\\
  \hline
\end{tabular}
%--------------------------------------------------

\section{namelists\_nampp.dat}
\label{sec:namelists_namppp}

This namelist is preferable created by the provided tool, \verb+SRC_DIR/tools/prepare_namelists_namppp.sh+
Input files for \verb+prepare_namelists_namppp.sh+ are available under

\verb+SRC_DIR/reference_setups/output_variables+. If you create your own input
file please follow the instructions in \verb+prepare_namelists_namppp.sh+.

\subsection{nampp}
\begin{tabular}{|l|l|l|l|l|}
  \hline
  varible & type[(length)] & explanation & default\\
  \hline
  \hline
  nppstr & int & number of output files & -1 \\
  \emph{suff} & char[2][10] & suffixes to files & 'pp','dd','qq','hh','ss','gb','  ' \\
  \emph{month\_file} & bool & write monthly output & .true. \\
  \hline
\end{tabular}

\subsection{namppp}
This namespace MUST be repeated \verb+nppstr+ (from nampp) times.
Each variable as written to an output file must be defined as a GRIB code (code, type, level)
which in the name list corresponds to (\verb+iwmoslp+, \verb+ltypslp+, \verb+alevslp+).
For a specification of available variables (\verb+iwmoslp+, \verb+ltypslp+, \verb+alevslp+) please refer to
Appendix \ref{app:gribcodes}.

\begin{tabular}{|l|l|l|l|l|}
  \hline
  varible & type[(length)] & explanation & default\\
  \hline
  \hline
  lunppfp & int &  unit number to write & \\
  \emph{prefixp} & char(2) & prefix for filename & 'fc' \\
  sufixp & char(2) &  suffix for filename & \\
  \emph{timeppp} & int &  output interval in seconds & \\
  nlevmlp & int & number of multi-levels  & \\
  nwmomlp & int & number of components of multi-level variables   & \\
  ltypmlp & int & grib-code  type for multi-level variable & \\
  alevmlp & int & grib-code level for multi-level variable & \\
  iwmomlp & int &  grib-code parameter for multi-level variable & \\
  nslp & int  & number of single levels & \\
  ltypslp & int(nslp) &  grib-code  type for single-level variable  & \\
  alevslp & real(nslp) &  grib-code level for single-level variable & \\
  iwmoslp & int(nslp) &  grib-code parameter for single-level variable  & \\
 \hline
\end{tabular}


\section{Lateral boundaries from a general circulation model}

This manual mainly describes how to use RCA with ERA40 or ERA-INTERIM, 
the re-analyses from ECMWF.
Suppose you want to run a climate scenario with lateral boundaries
from some global model. What input is necessary?

\begin{verbatim}
The 3-D fields needed are:
   temperature
   wind (u and v)
   specific humidity
\end{verbatim}

Preferably at model levels.
If only pressure levels are available, you have to interpolate to model levels.

\begin{verbatim}
You also need:
   sea surface temperature
   ice cover
   surface pressure
\end{verbatim}

\begin{verbatim}
and constant fields:
   land-sea mask
   orography (surface geopotential)
\end{verbatim}

\begin{verbatim}
For initialising the soil scheme you need some soil variables at different layers:
   soil moisture
   soil temperature
\end{verbatim}

An alternative to initialising the soil scheme is to run a longer spin-up.

All models are different, they use different formats, use different grids,
save different variables. So, retrieving what you need is different every time.
Therefore we can not give you one recipe for the work.

\begin{verbatim}
For example you might have to:

   change from NetCDF format to GRIB format
   change or set GRIB codes
   change from spectral coefficients to regular lat/long grid
   change from Gaussian grid to regular grid
   change from vorticity and divergence to U and V winds
   change from logarithm of surface pressure to surface pressure
   change units
   swap north/south to south/north
   extract a geographical area (to save space)
   split a file with data for one month into files with data for one time-step
\end{verbatim}

These things you can do inside RCA or pre-process data before running RCA.
We generally do it in a pre-processor.

If the grid is not global, it should be big enough for the regional grid
you will use.

RCA needs ASIMOF-files. ASIMOF format is close to GRIB format
but somewhat different. ASIMOF has a header in the beginning of the file.
Compilation of RCA also creates a tool that converts from GRIB to ASIMOF.
It is called pgb2as.x and is found in \verb+SRC_DIR/ARCH/bin+
\newline
The use of it is shown in appendix \ref{app:Prepareboundaries}.

Appendix \ref{app:Prepareboundaries}
shows some excerpts from a script that does some of the things mentioned above.
It uses CDO (https://code.zmaw.de/projects/cdo) and MARS from ECMWF.
If you don't have MARS, we are pretty sure you can do the same thing
with CDO or any tool that you prefer.


\section{Defining a grid, the geographical area}
In \verb+namelists.dat+ you have to edit the \verb+domain+ namelist to
define your computational domain.

Here you specify the southwestern corner \verb+south+ and \verb+west+. And the grid
distance in degrees \verb+dlon, dlat+. 
If you want to rotate the grid, set latitude and longitude for the
south pole \verb+polon, polat+.

\begin{verbatim}
Example:
&domain
        klon_global=134
        klat_global=155
        klev_global=40.
        south=-25.0
        west=159.5
        dlon=0.50
        dlat=0.50
        polon=110.0
        polat=-55.0
&end
\end{verbatim}

If you don't use the iterartive Helmholtz-solver (in namsl) there are restrictions in the choice of \verb+klon_global+.
The \verb+klat_global+ you may choose freely. This is due to the FFT-transforms to solve the Helmholtz-equation in longitudal direction.  
\begin{equation}
  klonGlobal = (2^n  3^m  5^k) + 2\cdot \underbrace{npbpts}_{\textrm{given in nambc}} +2 ,~~n>0~\textrm{and}~m,k\in\mathbf{Z}.
\end{equation}

A list of allowed \verb+klon_global+ is found in appendix
\ref{app:AllowedNLON}.

Currently RCA supports using
\verb+klev_global=19,24,31,40,50,60,62,91+.



\section{Version control for RCA using SVN}
Version control systems are designed to help keep track of documents
that are frequently revised by one or many authors. This is typically
the case when programming, and when producing a manuscript for a
scientific publication. Using a version control system makes your life
easier when 
\begin{itemize}
\item there are more than one person working on the project
\item you want to keep old versions of the code for future reference
  (and reversion) but find it silly to manually create directories
  like \verb+./code_v_0.1_backup_date+. 
\item you are writing a manuscript for a scientific publication
  together with collaborators 
\item you want to share the code with someone (i.e. your
  professor). You will want to share the latest working version, and
  not the messy directory that contains the current version of the
  code (which may not always work) along with all the junk that
  accumulates when you code, run, and test.  
\end{itemize}
In fact, a version control system will make it much more likely that
you manage to maintain control of the code. Version control makes you
a better programmer -- and it's not hard at all to use! This is a
simple tutorial on how to get started with Subversion (SVN), a modern
version control system. Visit the official homepage
\verb+http://subversion.tigris.org/+.
\subsection{Introduction to SVN}
There is a phenomenal book on SVN freely available online,
\verb+http://svnbook.red-bean.com/+. You
may read it and skip this tutorial completely (but the tutorial is not
300 pages). Either way, you will want to turn to it as a reference for
more involved stuff. Also, the first 40 pages give a very nice
introduction.  
\subsection{Check out a working copy}
Go to the place in the file system where you write code, such as
\verb+~/workspace/+. To check out the newly created repository do  
\begin{verbatim} 
user@host:~/workspace/ svn co svn+ssh://$USER@gimle.nsc.liu.se/home/rossby/svn/rca_repository/rca/trunk SRC_DIR
\end{verbatim}
It should say the it has checked out the latest revision (some
number). Now you have a working copy in the directory
\verb+project_name+. 

\subsection{Add files}
Go to the working copy. Put some code in this directory, say hello.c
and Makefile. Now put these files under version control by  
\begin{verbatim}
svn add hello.c Makefile
A         hello.c
A         Makefile
\end{verbatim}
The A above indicates "add", i.e. those files will be added to the
repository when the changes are committed.  
\subsection{Committing changes}
The final step is to commit the changes made (i.e. that two files have
been added). The command is  
\begin{verbatim}
user@host:~/workspace/ svn commit -m "Added files hello.c and Makefile"
Adding         hello.c
Adding         Makefile
Transmitting file data ........................
Committed revision X.
\end{verbatim}
\subsection{Tools for common tasks}
Always do svn help when you don't remember a command. Here are the
most useful ones: 
\begin{itemize}
\item \verb+svn status+ shows the status of the working copy. If we did this
  after adding the files above SVN would tell you that these
  two files are to be added to the repository (the A in the
  left column). File may otherwise be modified (M), deleted
  (D) or not under version control (?). 
\item \verb+svn status -u+: Stat against latest revision in the repository 
\item \verb+svn update+ updates against the latest revision in the repository.
\item \verb+svn diff "file"+ runs diff between this file and the one checked out from the repository. This is the convenient way to check what changes you have made to a particular file.
  \item \verb+svn diff -r HEAD+: runs diff against the lates revision on the repository. 
\end{itemize}
\subsection{Resurrecting a file from an older revision}

Say that the file foo.c has been removed, but now turns out to be
interesting again. How to get it back? Well first we have to know in
which revision it existed. We use the copy command in SVN: 
\begin{verbatim}
svn cp -r 100 file://$HOME/svnroot/repository_name/foo.c foo_old.c
\end{verbatim}
where the flag -r 100 specifies that we want the version of the file
that existed in revision 100. Note that this works equally well for
reverting a file that exists in the repository back to a previous
revision.  
\subsection{Some things NOT to do}
\begin{itemize}
    \item Do not move or remove files that are under version control
    using mv or rm, since this will break the version control of the
    code. Use svn move and svn rm instead. Note that files that have
    been removed can be reverted from earlier revisions. 
    \item Do not commit code to the repository that does not work. The
    repository should contain the latest working copy (in the sense
    that it runs at least). Fixing bugs, adding features etc, however
    is part of the development cycle (i.e. the code that is committed
    doesn't need to be "final").  
\end{itemize}
\subsection{RapidSVN - A GUI for SVN operations}
Subversion has many more features than what has been presented
here. And in certain circumstances problems occur. Two things to
remember when things look complicated are 
\begin{itemize}
\item Subversion has features that handle most conceivable problems that can occur in a development project (it's mostly a question of finding help).
\item The on-line guide has most answers (link above). 
\end{itemize}
Still, it can be comforting and convenient to have a GUI that gives an
overview over repositories, working copies, patches and
change-sets. Support for SVN is embedded in certain IDEs, but there is
a stand-alone tool called RapidSVN \verb+http://rapidsvn.tigris.org/+
that gets the complicated jobs done. 

\section{Oasis}
When coupling RCA to other codes we support the use of OASIS.
Oasis is downloaded by issuing:
\begin{verbatim}
svn checkout svn://memphis.cerfacs.fr/home/oasis/PRISMSVN/trunk/oasis4 oasis
\end{verbatim}
in the \verb+SRC_DIR+. \emph{To be continued...}

\section{LPJ-GUESS}
To check out LPJ-GUESS administer:
\begin{verbatim}
svn co svn://stormbringer.nateko.lu.se/svn/LPJ-GUESS/trunk
\end{verbatim}


%--------------------------------------------------
\appendix
\newpage
\section{Code structure}

This code structure represents the path through the code given \verb+lecmwf=.true.+ in the namelist \verb+namgcm+.

\verbatiminput{rca4_tree.txt}
%--------------------------------------------------
\newpage
\section{Prepare boundaries }
\label{app:Prepareboundaries}

\verb+# for CLARIS/South America:+
\newline
south=-60.
\newline
west=-110.
\newline
north=30.
\newline
east=0.
\newline
echo "southern America area"

\verb+indir=/nobackup/rossby14/sm_uhans/DATA/ECHAM5/Global/20C_1+
\newline
\verb+dirgrib=/nobackup/rossby14/sm_uhans/DATA/ECHAM5/Regional/GRIB+
\newline
\verb+prefixm=31009_S09_+
\newline
\verb+prefix6=S09_+
\newline
\verb+mkdir -p $dirgrib+

\verb+tmpdir=/scratch/local/rossby/sm_uhans/4RCA_EH5+
\newline
\verb+mkdir -p $tmpdir+
\newline
\verb+mkdir $tmpdir/splitteddays+
\newline
\verb+mkdir $tmpdir/splittedhours+

\verb+# one month at a time+

yyyy=1958
\newline
\verb+while (($yyyy <= 2000))+
\newline
do
\newline
for mm in 01 02 03 04 05 06 07 08 09 10 11 12
\newline
do

\verb+ln -s $indir/$prefixm$yyyy$mm.01 infil+

\begin{verbatim}
# get vo div from spectral file
#-----------------------------------------------------
cat > mars_in <<@@
read,
    source     ="infil",
    param      =vo/div,
    fieldset   ="fs"
write,
    fieldset   ="fs",
    target     ="vodivsh"
@@

mars  mars_in > /dev/null
\end{verbatim}

\verb+# make u v with cdo+
\newline
\verb+#-----------------------------------------------------+
\newline
cdo dv2uv vodivsh uvfromcdo

\begin{verbatim}
# extract area and interpolate u v 
#-----------------------------------------------------
cat > mars_in <<@@
read,
    source     ="uvfromcdo",
    grid       =1.875/1.875,
    area       =$north/$west/$south/$east,
    fieldset   ="fs"
write,
    fieldset   ="fs",
    target     ="uvml1"
@@

mars  mars_in > /dev/null
\end{verbatim}

\begin{verbatim}
# change codes
#-----------------------------------------------------
cdo selcode,131 uvml1 uml
cdo selcode,132 uvml1 vml
cdo setcode,33 uml u
cdo setcode,34 vml v
cdo merge u v uvml
\end{verbatim}

\begin{verbatim}
# splitting in 6-hour files
#-----------------------------------------------------
cd splitteddays
cdo splitday ../$prefix6$yyyy$mm $prefix6$yyyy$mm
cd ..
cd splittedhours

for dd in 01 02 03 04 05 06 07 08 09 10 11 12 13 14 15 16 17 18 19 20 21 22 23 24 25 26 27 28 29 30 31
do

cdo splithour ../splitteddays/$prefix6$yyyy$mm$dd.grb $prefix6$yyyy$mm$dd

for hh in 00 06 12 18
do

mv $prefix6$yyyy$mm$dd$hh.grb $prefix6$yyyy$mm$dd$hh"00+000H00M"

done #hh

done #dd

mv * $dirgrib
cd ..
rm splitteddays/*
rm $prefix6$yyyy$mm 

done #mm

yyyy=`expr 1 + $yyyy`
done #yyyy
\end{verbatim}

\begin{verbatim}
#-----------------------------------------------------
# converting from GRIB to ASIMOF
#-----------------------------------------------------

dirgrib=/nobackup/rossby13/sm_uhans/DATA/ECHAM5/GRIB	
dirasim=/nobackup/rossby14/sm_uhans/DATA/ECHAM5/ASIMOF/A1B_3
prefix6=S24_

YY=2001
while (( $YY <= 2101 ))
do
for MM in 01 02 03 04 05 06 07 08 09 10 11 12
do
for JD in 01 02 03 04 05 06 07 08 09 10 11 12 13 14 15 16 17 18 19 20 21 22 23 24 25 26 27 28 29 30 31
do
for JH in 00 06 12 18
do

ln -s $dirgrib/$prefix6\_$YY$MM$JD$JH\00+000H00M fort.1
ln -s $dirasim/$prefix6\_$YY$MM$JD$JH\00+000H00M fort.2
/home/sm_uhans/hirlam/$SRC_DIR/$ARCH/bin/pgb2as.x > /dev/null
rm fort.1
rm fort.2

done #JH
done #JD
done #MM
YY=`expr 1 + $YY`
done #YY

\end{verbatim}

%--------------------------------------------------
%
% This list was made at smhi:/smhi/home/Ulf.Hansson/mkgrid
% by the program nxny_fuj.f
%
\newpage
\section{Allowed klon\_global}
\label{app:AllowedNLON}
\begin{verbatim}
                 allowed 
  x  y  z klon_global-6 klon_global
  2  1  0     12           18
  4  0  0     16           22
  1  2  0     18           24
  2  0  1     20           26
  3  1  0     24           30
  1  1  1     30           36
  5  0  0     32           38
  2  2  0     36           42
  3  0  1     40           46
  4  1  0     48           54
  1  0  2     50           56
  1  3  0     54           60
  2  1  1     60           66
  6  0  0     64           70
  3  2  0     72           78
  4  0  1     80           86
  1  2  1     90           96
  5  1  0     96          102
  2  0  2    100          106
  2  3  0    108          114
  3  1  1    120          126
  7  0  0    128          134
  4  2  0    144          150
  1  1  2    150          156
  5  0  1    160          166
  1  4  0    162          168
  2  2  1    180          186
  6  1  0    192          198
  3  0  2    200          206
  3  3  0    216          222
  4  1  1    240          246
  1  0  3    250          256
  8  0  0    256          262
  1  3  1    270          276
  5  2  0    288          294
  2  1  2    300          306
  6  0  1    320          326
  2  4  0    324          330
  3  2  1    360          366
  7  1  0    384          390
  4  0  2    400          406
  4  3  0    432          438
  1  2  2    450          456
  5  1  1    480          486
  1  5  0    486          492
  2  0  3    500          506
  9  0  0    512          518
  2  3  1    540          546
  6  2  0    576          582
  3  1  2    600          606
  7  0  1    640          646
  3  4  0    648          654
  4  2  1    720          726
  1  1  3    750          756
  8  1  0    768          774
  5  0  2    800          806
  1  4  1    810          816
  5  3  0    864          870
  2  2  2    900          906
  6  1  1    960          966
  2  5  0    972          978
  3  0  3   1000         1006
 10  0  0   1024         1030
  3  3  1   1080         1086
  7  2  0   1152         1158
  4  1  2   1200         1206
  1  0  4   1250         1256
  8  0  1   1280         1286
  4  4  0   1296         1302
  1  3  2   1350         1356
\end{verbatim}

%--------------------------------------------------
\newpage
\section{GRIB codes}
\label{app:gribcodes}

Each variable is specified using its uniqe GRIB code (code, type, level) which 
corresponds to the first three columns in the list below. In general
``code'' refers to the character of the varibel (e.g. 11 for temperature, 33 for u-wind component), 
``type'' refers to which type the variable represent (e.g. 105 for land surface, 109 for model level,
100 for pressure level) and 
``level'' refers to which level the vaiable represent (.e.g. 0 for surface, 24 for model level 24,
850 for pressure level 850 hPa).
However, a number of exceptions to these general rules exist. For example,
``level''=3006 refers to mean value of the variable over the output interval (e.g. 111 105 3006),
``level''=4006 refers to accumulated value of the variable over the output interval (e.g. 62 105 4006).

In addition Rossby Centre has violated the GRIB standard by defining arrays of varaibles defiend by their code
number but where every unique variable in the array is defiend by its level. The code numbers for these arrays
are defined in the following table (all with type=105). The number of levels for each array are defined
by parameters in \verb+src/gemini.F90+. The number of levels for 252, 250, 243 and 244 must correspond to
the same number of variables
in \verb+call phys+ and in \verb+subroutine phys+.

%%%%%%%%%%%%%%%%%%%%%%%%%%%%%%%%%%%%%%
\begin{tabular}{l p{7cm} l}
{\bf code}	& {\bf description}							& {\bf parameter in gemini.F90}\\
\hline
252		& prognostic land and se-ice varibels					& \verb+ksvars+ \\
250		& for diagnostic instantaneous land and sea-ice variables		& \verb+msvars+ \\
242		& diagnostic mean values (over ecah specific output interval) of land and sea-ice variables & \verb+msvars+	\\
245		& diagnostic accumulated values (over ecah specific output interval) of land and sea-ice variables & \verb+msvars+ \\
241		& extreme time-step values (over ecah specific output interval) & \verb+esvars+ \\
243		& prognostic FLake variables 						& \verb+lake_no_prog+ \\
244		& diagnostic FLake variables 						& \verb+lake_no_diag+ \\
245		& ECOCLIMAP physiography variables					& \verb+meco+ \\
\hline
\end{tabular}
%%%%%%%%%%%%%%%%%%%%%%%%%%%%%%%%%%%%%%


{\bf List of variables:}

\verbatiminput{../../tools/rca35name.txt}
%--------------------------------------------------

\end{document}
